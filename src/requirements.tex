\clearpage
\section{Requirements Specification}

The primary objective of this project is to analyze, implement, and evaluate selected software-based fault tolerance techniques. The focus is on embedded systems operating in environments where faults can compromise system behavior. These issues are especially relevant in resource-constrained systems where safety and reliability are paramount, such as the aerospace or the automotive industry.
This section outlines the functional and non-functional requirements that the proposed system must fulfill to meet its objectives.

\subsection*{Functional Requirements}
\begin{itemize}
\item The system shall implement multiple software-based fault tolerance techniques.
\item The system shall be capable of detecting and reporting transient faults at runtime.
\item The system shall provide a mechanism for recovering from detected faults where applicable.
\end{itemize}

\subsection*{Non-Functional Requirements}
\begin{itemize}
\item The system shall introduce minimal performance overhead relative to the unprotected baseline.
\item The solution shall be implemented entirely in software using the Rust programming language.
% \item All fault tolerance methods shall be implemented in the Rust programming language to leverage its memory safety guarantees and low-level control.
\item The system shall be portable and capable of running on a 32-bit RISC-V platform using the FreeRTOS operating system.
\item The implementation shall be maintainable and modular, enabling future extension or substitution of fault tolerance mechanisms.
\end{itemize}

Ultimately, the project aims to deliver a practical and reproducible evaluation framework for software-based fault tolerance, enabling developers to assess the trade-offs between fault tolerance, runtime overhead, and system complexity in embedded contexts.