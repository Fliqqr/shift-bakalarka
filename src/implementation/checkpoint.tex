\subsection{Checkpoint and restart}

Checkpoint and restart system is the backbone of our fault-tolernace framework, facilitating recovery. It can be used on its own to provide basic level of redundancy, or it can be combined with other techniques to boost their fault-tolerance potential.

\subsubsection{Creating a checkpoint} \label{sec:creating_checkpoint}

\begin{figure}[!h]
\begin{lstlisting}[language=Rust]
asm!(
    // Storing registers in a static variable
    "lui   t0, %hi(CHECKPOINT)",
    "addi  t0, t0, %lo(CHECKPOINT)",
    "sw x1,   4(t0)",
    "sw x2,   8(t0)",
    ...
    "sw x31,124(t0)",
    "call set_checkpoint",
    // Restoring registers to their previous state
    "lui   t0, %hi(CHECKPOINT)",
    "addi  t0, t0, %lo(CHECKPOINT)",
    "lw x1,   4(t0)",
    "lw x2,   8(t0)",
    ...
    "lw x31,124(t0)",
)
\end{lstlisting}
\caption{Rust - Creating checkpoint}
\label{fig:rust_create_checkpoint}
\end{figure}

Checkpoint is generated by creating a \textit{checkpoint context} during which the current state of the application is stored. Namely, the general purpose registers x1 - x31 in RISC-V 32bit (x0 is ommitted as its hardwired to always be 0, the hardware implementation is resopnsible for ensuring this and as such this register cannot be written to) are stored into a static memory location before the execution of a \textit{checkpoint block} (the part of code protected by the checkpoint). Additionally, a \textit{checkpoint address} (address of the first recovery instruction, Fig. \ref{fig:rust_create_checkpoint} line 11) is stored along with the general purpose registers. If we directly used the saved return address (x1) to jump back after an exception, the control would be transferred to outside our checkpoint context. 

Higher-level programming languages do not directly provide the functionality to manipulate register values. This means assembly instructions have to be used to facilitate the low level access as seen in figure \ref{fig:rust_create_checkpoint}.

In order to set the checkpoint address, we utilize a trick (shown in figure \ref{fig:rust_set_checkpoint}) whereby calling a function we now have access to the address of the instruction immediatelly following the function call (figure \ref{fig:rust_create_checkpoint} line 11) saved in the return address register (RA). We store the value of this register along with the registers in the checkpoint variable.

\begin{figure}[!h]
\begin{lstlisting}[language=Rust]
fn set_checkpoint() {
    unsafe {
        asm!(
            "sw ra, ({0})",
            in(reg) &(CHECKPOINT.ret_addr),
        );
    }
}
\end{lstlisting}
\caption{Rust - Set checkpoint}
\label{fig:rust_set_checkpoint}
\end{figure}

\subsubsection{Storing the checkpoint}

As mentioned in section \ref{sec:creating_checkpoint}, the checkpoint information is stored in a static memory location. This might seem as an arbitrary decision given that storing the checkpoint on the stack or the heap are also an option. Each of the listed approaches comes with its own issues. Using a static memory location limits the number of checkpoints that can be stored. Using the stack for checkpoint storage can be very unreliable as stack corruption can easily occur should the stack pointer manifest a fault. And using the heap is generally slow and prone to error stemming from pointer corruption or premature memory freeing which could occur in the presence of faults.

Our implementation uses a hybrid approach of using a static memory location for the storage of the \textit{primary checkpoint} (the last checkpoint that was set), while pushing any older checkpoints on the stack. This heightens the chance that at least the primary checkpoint will be available. Since a static memory location is used, we could also store the primary checkpoint in a hardware-hardened memory section. Once the primary checkpoint context ends, the previous checkpoint is popped from the stack and can be used again. This allows for infinite nesting of checkpoints, provided we have enough memory.

\subsubsection{Returning to the checkpoint}

After setting the checkpoint, the program execution continues as per usual, until either the checkpoint block finishes successfuly, or an error is detected. An error can be detected in two way, an exception is detected by the processor, or an error is detected via checks inserted into the checkpoint block.

If the processor detects an error the exception handler subroutine is called. We can hook into this subroutine and use it to restore the stack pointer to the pre-checkpoint block state and then perform a jump back to the saved checkpoint address.

If an error is detected via checks inserted by the programmer, checkpoint restart can be triggered by simply returning an error (Err()) from within the checkpoint block.

Both forms of error detection and returning to checkpoint result in the restoration of the program context to the program state in which the checkpoint function was called. After the context was restored, checkpoint block will be attempted again if retries are allowed (retries parameter != 0, see Fig. \ref{fig:rust_using_checkpoint} - retries is set to 2), otherwise the entire checkpoint context returns an erroneous value.

\begin{figure}[!h]
\begin{lstlisting}[language=Rust]
if let Err(Error::MaxRetriesReached(n)) = checkpoint(2, || {
    // Checkpoint block here
    ...
    Ok(())
}) {
    println(&format!("Max retries reached: {}", n));
}
\end{lstlisting}
\caption{Rust - Using the checkpoint system}
\label{fig:rust_using_checkpoint}
\end{figure}

\subsubsection{Overhead}

The implemented checkpoint system incurs minimal instruction overhead. Additional instructions are only inserted when creating the checkpoint context and just before the end of the checkpoint context. This comes out to 79 additional instructions to save and restore program state, plus approximately 300 instructions to save and restore old checkpoint and perform various error checks, without factoring in any compiler optimizations.

This overhead is constant for the entire checkpoint block, irrespective of the checkpoint block size. As such, the ratio of overhead instructions to the checkpoint block instructions is solely determined by the size of the protected checkpoint block.

\subsubsection{Considerations} \label{sec:checkpoint_considerations}

Checkpoint system only facilitates basic restart and retry functionality. It can be used on its own when only basic level of protection is required, however, in most cases, checkpoint and restart system is meant to be used as a framework to build upon. Namely, the multi-version techniques outlined in later sections all use checkpoints to implement version switching.

Our implmenetation of the checkpoint and restart system only saves the minimal required state, which in the case of RISC-V, is the general purpose registers. This limits the use of the checkpoint system as other memory regions such as stack or heap are not stored. Careful consideration needs to be given to the code block protected by the checkpoint context. If the checkpoint body has any side-effects, such as modifying global variables, variables outside of the checkpoint block scope or writing to heap, these alterations will not be reversed in the case of an error and subsequent checkpoint return.

This problem can be avoided by explicitly creating a copy of any outside variable which needs to be modified within the checkpoint block. The copy will then be used within the checkpoint block and the original variable will only be overrided if the checkpoint context returns successfully. More on this in section \ref{sec:var_protection}.

Another important aspect to mention is that our implementation does not save registers other than the general purpose registers. This means special registers such as control and status registers (\textit{mtvec}, \textit{mcause} or \textit{mstatus}) are not protected by the checkpoint system. The fault-insertion simulation does not insert faults into these registers, as such their directly protection is not needed. However, errors can still manifest (as if they occuerd in the special registers) in ALU after the special registers are read for processing.
